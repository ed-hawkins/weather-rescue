\documentclass[12pt, epsfig, graphics, rotating, epsf,tifff]{article}
\usepackage{float}
\usepackage{lscape,epsfig,portland,portland,setspace,rotating,subfigure,pifont}
\usepackage{graphicx}
\pagestyle{empty}

\oddsidemargin  0pt
\evensidemargin 0pt 
\marginparwidth 10.6mm
\marginparsep 1pt
\setlength{\parindent}{0pt}
\setlength{\topmargin}{-5.4mm}
\setlength{\textwidth}{150mm}
\setlength{\textheight}{240mm}
\setlength{\oddsidemargin}{1.6mm}
\setlength{\evensidemargin}{10.6mm}
%\includegraphics{map_obs2.gif} 
 
\begin{document}



\begin{center}

\vspace{-5mm}\LARGE{\bf Meteorological Data recorded at Armagh Observatory:
Vol 3 - Meta-data, 1796-2006}

\vspace{5mm}

\normalsize
{A. M. Garc\'{i}a-Su\'{a}rez, E. Park, C.J. Butler, K. Hickey, A. Grant}

%\vspace{5mm}

{\em Armagh Observatory, College Hill, Armagh BT61 9DG, N. Ireland}
\end{center}
	
\vspace{-0.5cm}
\section{Introduction}

Armagh Observatory has the longest series of meteorological records from any
single site in Ireland and one of the longest in the British Isles. Pressure
and air temperature readings commenced in 1795, to be joined in 1838 by
rainfall and wet and dry temperatures and in 1843 by maximum and
minimum temperatures. Records of bright sunshine from a
Campbell-Stokes sunshine recorder commenced in 1880 and soil temperatures at
one foot and four foot depth in 1904. In the late 18th century and early 19th
century, readings were made thrice daily, in the morning, at midday and in the
evening. Later, this was reduced to twice daily morning and evening readings
and finally from 1966, to just morning readings.

\medskip

Up until June 2000, the data have been entered into the manuscript meteorological 
record books (M117, Butler and Hoskin, 1987) with monthly returns to the UK Met
Office since 1868. These manuscript books have been scanned electronically and
the images made available to the general public via the World-Wide Web. Later
data is available in digital format.

\medskip

In addition, the main data series, namely: air pressure, air temperature,
sunshine, humidity and soil temperatures have been entered into computer file
and, where possible, calibrated and corrected for instrumental error and
exposure. These data sets (see list in references below) are to be made
available over the Internet at http://climate.arm.ac.uk for scientific research
and educational purposes and will be updated as new data accumulates. Printed
versions for limited circulation are available on request from {\em The
Librarian, Armagh Observatory}.

\section{Metadata}

In order to make the required corrections for instrumental error and exposure,
detailed knowledge is required of the history of the instruments and any tests
on their accuracy before and during their period of use. Such information is
termed {\em metadata}. It encompasses all information relevant to the
instruments, their location and use, and includes such items as: letters to and
from observers, details of observation times and practices, vouchers and
receipts, details of procedures, inspection reports, maps, drawings and
photographs. Some of these documents (assigned the prefix M) have been
previously listed in the more general catalogue of historical Observatory
documents by Butler and Hoskin (1987). 


\medskip

\singlespace
%\begin{table}[!h]
\begin{table}[h!]
\vspace{-0.8cm}       \caption[ ]
{\footnotesize{Key for the contents or topics in the documents.}}
\label{key}
\begin{center}
\vspace{-0.4cm}\scriptsize{

\begin{tabular}{|l|l||l|l|}

\hline
Topic No. & Topic&Topic No. & Topic\\
\hline

1 & Barometric pressure                                &21&  Photographs of met. site and instruments\\
2 & Temperature 				       &22&  Publications				       \\
3 & Wind					       &23&  Tornadoes and waterspouts\\
4 & Rainfall					       &24&  Solar link and sunspots  \\
5 & Sunshine					       &25&  Flooding		\\
6 & Cloud					       &26&  List of inspectors \\
7 & Humidity					       &27&  Funding of meteorological work\\
8 &  Fog					       &28&  War correspondence 1939-1945  \\
9 &  Visibility 	      			       &29&  Staff	  \\
10&  Thunder and lightning    			       &30&  Obituaries   \\
11&  Snow, hail, frost  	     		       &31&  Observers, contributors  \\
12&  Recording forms and instructions		       &32&  Phenological data \\
13&  Reports on station 			       &33&  Stationery 	       \\
14&  Instrument list				       &34&  List of inspectors reports\\
15&  Plans of observatory met. station       &35&  Requests for publications\\
16&  Request for information			       &36& Statistics  	    \\
17&  Newspaper articles 			       &37& Summary/outline of project     \\
18&  Reference to old/missing data	       &38& Data, codes, format, files \\
19&  Questionnaire		    		        &39& Validation of data, calibration, etc  \\
20&  Self recording station (hourly)		       &40& Climatology, weather		  \\
						       


\hline
   \end{tabular}
}
\end{center}
\end{table}			

\medskip


In  Table 3, the documents listed are in chronological order. Column 1 gives the document
reference number, column 2 the number in the list by Butler and Hoskin (1987)
if it appears there, column 4 the date if known, column 5 code numbers relevant
to the context of the document and lastly, column 6 a short description of the
contents. Currently the catalogue contains over a thousand documents and many
more will be included in future revisions. Though each document has only a
single reference number in this list, they vary widely in size from a single
page to over 10,000 pages. In Table 1 we give the list of code numbers and the
context in which the documents are useful as metadata for the meteorological
data series. Table 2 gives some examples. In Table 4, we give the same
information as in Table 3, but in document numerical order.

\medskip

 A document can be searched in a number of ways in the catalogue: by  catalogue
 number, by date and/or by code number (following the key listed in
 Table~\ref{key}).  Note that each document can have several code numbers. For
 example, the document  ARM/MET/000140	(in 1952) is an inspector's report
 which lists corrections and the  equipment for several meteorological
 parameters. The code numbers are: 1, 2, 3, 4, 5, 6, 13, 14 which refer to
 barometric pressure, temperature, wind, rainfall, sunshine, cloud,  reports on
 station and an instrument list respectively.





\singlespace
%\begin{table}[!h]
\begin{table}[h!]
\vspace{-0.3cm}       \caption[ ]
{\footnotesize{Structure of the catalogue and some examples}}
\label{sample}
\begin{center} 
\vspace{-0.7cm}\scriptsize{

\begin{tabular}{|@{  }l@{  }|@{  }c@{  }|@{  }l@{  }|@{  }l@{  }|@{  }l@{  }|}
\hline
Document Code & B$\&$H Code &  Date &	   Content&		   Description\\

\hline
ARM/MET/000590 &      &      2001  &          14 13 15     &       Armagh Obs. records and index. Met books at the library\\
ARM/MET/000591 & M120 &      1910-1919 &      1-10 22 36-40  &     M120 documents (copy) miscellaneous letters\\
ARM/MET/000592&		&   27/10/1999&	   13 12	&	   Letter inspection of Armagh climate site 1999.\\
ARM/MET/000593 &      &      8/4/2002	&   18 2 1-16	&	   Letter to D. Campbell and answer\\

\hline
   \end{tabular}
}\\B$\&$H Code is the Butler and Hoskin Code (in Butler and Hoskin, 1987).
\end{center}
\end{table}			

\vspace{-1cm}
 \section{Acknowledgements}
 
We wish to record our thanks to the Heritage Lottery Fund and the Irish Soldiers
and Sailors Land Trust for their financial support for this project. Research at
Armagh Observatory is grant-aided by the Department for Culture Arts and Leisure
for Northern Ireland.

\vspace{-0.3cm}
\section{References.}

Butler, C. J. and Hoskin, M. A. 1987. The archives of Armagh Observatory, J. Hist.
Astron. 18, 295-307. 


\medskip

\medskip

\noindent
{\bf Armagh Observatory Climate Series:\\ 
Meteorological Data Recorded at Armagh Observatory}

\medskip

\noindent
{\bf Vol 1} - Daily, Monthly and Annual Rainfall,
1838-2001. Garc\'{i}a-Su\'{a}rez' A.M., Butler, C.J., Cardwell, D, Coughlin, A.D.S., Donnnelly,
A., Fee, D.T., Hickey, K.R., Morrow, B. and Taylor, T. (2002)\\

\noindent
{\bf Vol 2} - Daily, Mean Monthly, Seasonal and Annual, Maximum and Minimum Temperatures
1844-2004, Butler, C.J., Garc\'{i}a-Su\'{a}rez A.M., Coughlin, A.D.S.,  and Cardwell, D.
(2005)\\

\noindent
{\bf Vol 3} - Meta-data, 1796-2006. Garc\'{i}a-Su\'{a}rez, A.M., Park, E., Butler, C.J., Hickey, K.R. 
and Grant, A. (2005)\\

\noindent
{\bf Vol 4} - Daily, Mean Monthly and Annual Soil Temperatures, 1904-2002.
Garc\'{i}a-Su\'{a}rez, A.M., Butler, C.J. and Morrow, B. (2005)\\

\noindent
{\bf Vol 5} - The Storminess Record, 1796-2002. Hickey, K.R.
(2005)\\

\noindent
{\bf Vol 6} - Daily, Monthly, Seasonal and Annual, Air Temperatures from Series I
(1796-1882), including the Dunsink Patch (1825-1833) and Series III (1844-1964).
Butler, C.J., Coughlin, A.D.S., Johnston, D.J., Cardwell, D. and Morrell, C. (2005)\\

\noindent
{\bf Vol 7} - Daily, Mean Monthly and Annual Wet and Dry Temperatures and
Relative Humidity. 1844-2003. Garc\'{i}a-Su\'{a}rez, A.M. and Butler, C.J., Coughlin,
A.D.S., Cardwell, D., Boyle, E., Ryan, G. and Dougan, L. (2005)\\

\noindent
{\bf Vol 8} - Hourly temperatures for the Self-Recording Thermograph of the Automatic
Weather Station at Armagh Observatory, 1874-1883. Grant, A., Garc\'{i}a-Su\'{a}rez,
A.M. and Butler, C.J. (2005)\\

\noindent
{\bf Vol 9} - Temperatures recorded at Dunsink Observatory, 1818-1850. Butler, C.J.,
Morrow, B. and Morrell, C. (2005)\\

\noindent
{\bf Vol 10} - Daily, Monthly and Annual Hours of Bright Sunshine, 1880-2004. Butler,
C.J., Emerson, M., Garc\'{i}a-Su\'{a}rez, A.M., Palle, E. and Kelly, S.T. (2005) \\

\noindent
{\bf Vol 11} - Twice-daily, Mean Monthly and Annual Pressure, 1796-2003. Butler,
C.J., Emerson, M., Allen, R., Ansell, T., Garc\'{i}a-Su\'{a}rez, A.M. (2005)\\

\noindent
{\bf Vol 12} - Unusual atmospheric phenomena recorded at Armagh Observatory,
1833-1941. Butler, C.J., Speers, J., Fee, D. and Dunne, E. (2006)\\


\end{document}
